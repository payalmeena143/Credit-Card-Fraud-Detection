% Chapter 4
\let\cleardoublepage\clearpage
 % \let\cleardoublepage\clearpage
\chapter{Results} % Main chapter title

\label{Chapter4} % For referencing the chapter elsewhere, use \ref{Chapter1} 

%----------------------------------------------------------------------------------------

% Define some commands to keep the formatting separated from the content 
\newcommand{\keyword}[1]{\textbf{#1}}
\newcommand{\tabhead}[1]{\textbf{#1}}
\newcommand{\code}[1]{\texttt{#1}}
\newcommand{\file}[1]{\texttt{\bfseries#1}}
\newcommand{\option}[1]{\texttt{\itshape#1}}



The culmination of the credit card fraud detection project brings us to the assessment of the Random Forest Classifier's and Logistic Regression  performance and the extraction of key findings, shedding light on the model's strengths, weaknesses, and overall efficacy in safeguarding against fraudulent transactions.

\section{Model Performance:-}
After meticulous training and fine-tuning, the Random Forest Classifier and Logistic Regression Classifier demonstrated commendable performance on the test dataset. The accuracy, precision, and recall metrics collectively affirm the model's ability to discern between legitimate and fraudulent transactions. The confusion matrix provides a detailed breakdown of true positives, true negatives, false positives, and false negatives, offering a comprehensive view of the model's predictive capabilities.The Logistic Regression Classifier give me high accuracy and all other scores comparison to Random Forest Classifier.

\section{ Key Findings:-} 

In the process of evaluating the model, several key findings emerged, providing valuable insights into the nature of credit card fraud and the efficacy of the Logistic Regression model. The feature importance analysis highlighted specific variables crucial in identifying fraudulent transactions.Additionally, patterns and trends within the dataset that contribute to successful fraud detection were identified. These key findings not only validate the model's performance but also contribute to a deeper understanding of the dynamics of credit card fraud, enabling continuous improvement and adaptation.

These results and key findings serve as the foundation for ongoing enhancements, ensuring that the credit card fraud detection system remains vigilant and effective in safeguarding financial transactions against emerging threats.

