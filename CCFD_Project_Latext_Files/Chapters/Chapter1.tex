% Chapter 1
\let\cleardoublepage\clearpage
 % \let\cleardoublepage\clearpage
\chapter{Introduction} % Main chapter title

\label{Chapter1} % For referencing the chapter elsewhere, use \ref{Chapter1} 

%----------------------------------------------------------------------------------------

% Define some commands to keep the formatting separated from the content 
\newcommand{\keyword}[1]{\textbf{#1}}
\newcommand{\tabhead}[1]{\textbf{#1}}
\newcommand{\code}[1]{\texttt{#1}}
\newcommand{\file}[1]{\texttt{\bfseries#1}}
\newcommand{\option}[1]{\texttt{\itshape#1}}

%----------------------------------------------------------------------------------------
\section{Background}
Credit card fraud is a persistent challenge in the financial landscape, with criminals constantly devising new methods to exploit vulnerabilities. This project aims to develop an effective credit card fraud detection system using advanced machine learning techniques.\medskip

The credit card has become the most popular payment method for both online and offline transactions. The necessity to create a fraud detection algorithm to precisely identify and stop fraudulent activity arises as a result of both the development of technology and the rise in fraud cases

\section{About}
There are various fraudulent activities detection techniques
has implemented in credit card transactions have been kept
in researcher minds to methods to develop models based on
artificial intelligence , data mining, fuzzy logic and machine
learning. Credit card fraud detection is significantly difficult,
but also popular problem to solve. In our proposed system
we built the credit card fraud detection using Machine
learning. With the advancement of machine learning
techniques. Machine learning has been identified as a
successful measure for fraud detection. A large amount of
data is transferred during online transaction processes,
resulting in a binary result: genuine or fraudulent. Within the
sample fraudulent datasets, features are constructed. These
are data points namely the age and value of the customer
account, as well as the origin of the credit card. There are
hundreds of features and each contributes, to varying
extents, towards the fraud probability. Note, the level in
which each feature contributes to the fraud score is
generated by the artificial intelligence of the machine which
is driven by the training set, but is not determined by a fraud analyst. \medskip

So, in regards to the card fraud, if the use of cards to
commit fraud is proven to be high, the fraud weighting of a
transaction that uses a credit card will be equally so.
However, if this were to shrink, the contribution level would
parallel. Simply make, these models self-learn without
explicit programming such as with manual review. Credit
card fraud detection using Machine learning is done by
deploying the classification and regression algorithms. We
use supervised learning algorithm such as Random forest
algorithm to classify the fraud card transaction in online or
by offline. Random forest is advanced version of Decision
tree. Random forest has better efficiency and accuracy than
the other machine learning algorithms. Random forest aims
to reduce the previously mentioned correlation issue by
picking only a subsample of the feature space at each
split. Essentially, it aims to make the trees de-correlated
and prune the trees by fixing a stopping criteria for node
splits, which I will be cover in more detail later.

%----------------------------------------------------------------------------------------
% \let\cleardoublepage\clearpage
\section{Problem Definition}

Billions of dollars of loss are caused every year by the
fraudulent credit card transactions. Fraud is old as humanity
itself and can take an unlimited variety of different forms.
The PwC global economic crime survey of 2017 suggests that
approximately 48% of organizations experienced economic
crime. Therefore, there is definitely an urge to solve the
problem of credit card fraud detection.Moreover, the
development of new technologies provides additional ways
in which criminals may commit fraud. The use of credit cards
is prevalent in modern day society and credit card fraud has
been kept on growing in recent years. Hugh Financial losses
has been fraudulent affects not only merchants and banks,
but also individual person who are using the credits. Fraud
may also affect the reputation and image of a merchant
causing non-financial losses that, though difficult to quantify
in the short term, may become visible in the long period. For
example, if a cardholder is victim of fraud with a certain
company, he may no longer trust their business and choose a
contender.

\section{Scope of the Project}

In this proposed project we designed a protocol or a model
to detect the fraud activity in credit card transactions.
This system is capable of providing most of the essential
features required to detect fraudulent and legitimate
transactions. As technology changes, it becomes difficult to track the
behaviour and pattern of fraudulent transactions.
With the upsurge of machine learning, artificial intelligence
and other relevant fields of information technology, it
becomes feasible to automate the process and to save some
of the effective amount of labor that is put into detecting
credit card fraudulent activities.\medskip

The scope of this project revolves around leveraging advanced machine learning techniques, specifically Random Forest and Logistic Regression, to fortify credit card fraud detection systems. The aim is to enhance the accuracy, efficiency, and adaptability of the current fraud detection methodology, addressing the challenges posed by imbalanced datasets and the ever-evolving nature of fraudulent activities.


















%----------------------------------------------------------------------------------------


%----------------------------------------------------------------------------------------






%----------------------------------------------------------------------------------------






